


If the resonant data sample can be divided based on the type of
parton that initiated the jets, the sensitivity of the search for some
resonances could be increased. 
The fraction of events as a function of  
the dijet mass from QCD  simulated with a  \textsc{Pythia~8.186}~\cite{pythia8} and the leading-order NNPDF2.3~\cite{Ball:2012cx} 
parton distribution functions (PDFs) is shown in Fig.~\ref{fig:quarkgluonfraction}. 
This suggests that tagging quark and gluon jets should be able to improve the sensitivity of searches for new particles. 
ATLAS has published a study \cite{ATL-PHYS-PUB-2017-009} showing that jets can be tagged as quark or gluon jets 
based on the number of charged particles with transverse momentum (\pt ) above 500\,MeV. 
In this note we present the analysis for the whole combined 2015 and
2016 $\sqrt{s} = 13$~\TeV\xspace dataset using quark 
and gluon tagging  based on charged-particle constituent multiplicity.


\begin{figure}[htb]
 \centering
\includegraphics[width=0.75\textwidth]{figures/tagging/Fractions_QCD.pdf}
\caption{The fraction of dijet events that are initiated by quark-quark events (blue), quark-gluon 
events (green) and gluon-gluon events (red) in simulated data.  \label{fig:quarkgluonfraction}}
\end{figure}

It has been shown~\cite{ATL-PHYS-PUB-2017-009}  that the number of charged particles with transverse
momentum (\pT ) above 500\,MeV, \ntrk, associated with a jet can be used
to select samples which have an increased fraction of jets produced by
quarks (or gluons) 
%Samples with enhanced fractions of quark or gluon
%initiated jets can be created by using a selection based on the \ntrk\
as shown in Fig.~\ref{fig:jet_pt_quark_gluon}.
Ref.~\cite{ATL-PHYS-PUB-2017-009} also shows that the \Pythia8
generator~\cite{pythia8} using the A14 tune~\cite{A14tune} is in a good
agreement with the distribution of \ntrk\ found in data.


\begin{figure}[htb]
 \centering
\includegraphics[width=0.75\textwidth]{figures/tagging/fig_01_ATL-PHYS-PUB-2017-009.pdf}
\caption{Distribution of the jet reconstructed track multiplicity (\ntrk ) in
 different pT ranges with the \Pythia~8 generator~\cite{pythia8} using
 the A14 tune~\cite{A14tune}, the NNPDF2.3 PDF
 set~\cite{Carrazza:2013axa}, and processes with a full simulation of the
 ATLAS detector. Jets must be fully within the tracking acceptance
 ($|\eta|<2.1$) and tracks are required to have $\pT > 500$\,MeV and pass
  quality criteria described in Ref.~\cite{ATL-PHYS-PUB-2017-009}. Figure
 from Ref.~\cite{ATL-PHYS-PUB-2017-009}. \label{fig:jet_pt_quark_gluon}}
\end{figure}


\subsection{Expected Signal Significance}
\label{sec:ExpectedSig}

The dijet mass spectrum has a complex background with rapidly changing  fractions of events that originate 
from quark-quark, quark-gluon and gluon-gluon scattering. The expected signal significance has been investigated using 
MC simulated  signals and background. The background is represented using the \QCD samples described 
in Section~\ref{qcdsamps}. 

\subsubsection{Signals that decay to quark-quark.}

The significance for signals that decay to a quark anti-quark pair are estimated using \Zprime\ models (Section~\ref{subsec:MCZprimeSamples}) using 
\begin{equation}
S = N_S \sum_i{ \dfrac{ f_{{qq}_i}\epsilon_{qQ}^2 + f_{{qg}_i}\epsilon_{qQ}\epsilon_{gQ} + f_{{gg}_i}\epsilon_{gQ}^2  } {\sqrt{ B_{{qq}_i}\epsilon_{qQ}^2 + B_{{qg}_i}\epsilon_{qQ}\epsilon_{gQ} + B_{{gg}_i}\epsilon_{gQ}^2  }}}
\end{equation}
where $N_S$ is the number of signal events, $f_{{qq}_i}$ is fraction of signal events that result in the two 
highest \pT\ jets that where initiated by quarks in bin $i$ ( $f_{{qg}_i}$ are quark-gluon jets, and $f_{{gg}_i}$ is two gluon jets), 
$\epsilon_{qQ}$ is the efficiency of a quark initiated jet passing the quark selection criteria, 
$\epsilon_{gQ}$ is the efficiency of a gluon initiated jet passing the quark selection criteria, 
and $B_{{xx}_i}$ is the expected number of background events with quark-quark, quark-gluon or gluon-gluon initiated jets. 


The significance is calculated for \Zprime's with masses ranging from 1500 to 4000\,\GeV and quark-jet selection 
efficiencies ranging from 30 to 90\%. The resulting significances are  in Fig.~\ref{fig:QuarkSignalSignificance}
and show that the significance decreases if any quark-selection is applied to the data. This is because
the data is dominated by quark-quark events and the selection reduces background and signal in similar amounts. 

\begin{figure}[htb]
 \centering
\includegraphics[width=0.75\textwidth]{figures/tagging/QuarkSignalSignificance.png}
\caption{ The significance for observing a \Zprime\ with masses from 1500 to 4000\,\GeV
for $\epsilon_{qQ}$ ranging from 90 to 30\% compared to the significance calculated with no quark selection applied. The key gives pairs of efficiencies ($\epsilon_{qQ}$, $\epsilon_{gQ}$).
  \label{fig:QuarkSignalSignificance}}
\end{figure}

\subsubsection{Signals that decay to gluon-gluon.}

The significance for signals that decay to a gluon-gluon pair are estimated using \Hprime\ models  using 
\begin{equation}
S = N_S \sum_i{ \dfrac{ f_{{qq}_i}\epsilon_{qG}^2 + f_{{qg}_i}\epsilon_{qG}\epsilon_{gG} + f_{{gg}_i}\epsilon_{gG}^2  } {\sqrt{ B_{{qq}_i}\epsilon_{qG}^2 + B_{{qg}_i}\epsilon_{qG}\epsilon_{gG} + B_{{gg}_i}\epsilon_{gG}^2  }}}
\end{equation}
where  
$\epsilon_{qG}$ is the efficiency of a quark initiated jet passing the gluon selection criteria, 
$\epsilon_{gG}$ is the efficiency of a gluon initiated jet passing the gluon selection criteria. 
The significance is calculated for \Hprime's with masses ranging from 2000 to 7000\,\GeV\ and quark-jet selection 
efficiencies ranging from 60 to 90\%. The resulting significances are  in Fig.~\ref{fig:GluonSignalSignificance}
and show that the significance increases from approximately 1.2 at 2\,TeV\ to 1.6 at 7\,\TeV\ with the greatest 
increases  occurring for gluon selection efficiency of 75\%. 

\begin{figure}[htb]
 \centering
\includegraphics[width=0.75\textwidth]{figures/tagging/GluonSignalSignificance.png}
\caption{ The significance for observing a \Hprime\ with masses from 2000 to 7000\,\GeV\ 
for $\epsilon_{gG}$ ranging from 90 to 60\% compared to the significance calculated with no gluon selection applied. The key gives pairs of efficiencies ($\epsilon_{gG}$, $\epsilon_{qG}$).
  \label{fig:GluonSignalSignificance}}
\end{figure}


\subsubsection{Signals that decay to quark-gluon.}


The calculation of the significance for a quark-gluon signal such as an excited quark decay is much more complicated as 
both jets can satisfy both the quark and the gluon selection.  In this case we need to define three efficiencies that are exclusive for quark and gluon jets. They are 
\begin{itemize}
\item $\epsilon_{qQ}$ The probability that a quark initiated jet is identified only as a quark jet. 
\item $\epsilon_{qQG}$ The probability that a quark initiated jet is identified  as a quark and a gluon jet.
\item $\epsilon_{qG}$ The probability that a quark initiated jet is identified only as a gluon jet.  
\end{itemize}
where by construction $\epsilon_{qQ} + \epsilon_{qQG} + \epsilon_{qG} = 1$. A similar set of efficiencies are measured 
for gluon initiated jets. 
\begin{itemize}
\item $\epsilon_{gQ}$ The probability that a gluon initiated jet is identified only as a quark jet. 
\item $\epsilon_{gQG}$ The probability that a gluon initiated jet is identified  as a quark and a gluon jet.
\item $\epsilon_{gG}$ The probability that a gluon initiated jet is identified only as a gluon jet.  
\end{itemize}

The probability of truth quark-quark ($p_{qq}$), quark-gluon ($p_{qg}$) and gluon-gluon ($p_{gg}$) truth 
events of passing the selection criteria are given by 
\begin{align}
p_{qq} & = 2  \epsilon_{qQ}\epsilon_{qG} + \epsilon_{qQG}\left( \epsilon_{qQ} + \epsilon_{qG} \right)  + \epsilon_{qQG}\epsilon_{qQG} \\
p_{gg} & = 2  \epsilon_{gQ}\epsilon_{gG} + \epsilon_{gQG}\left( \epsilon_{gQ} + \epsilon_{gG} \right)  + \epsilon_{gQG}\epsilon_{gQG} \\
p_{qg} & = \epsilon_{qQ}\epsilon_{gG} + \epsilon_{gQ}\epsilon_{qG} + \epsilon_{qQG}\left( \epsilon_{gQ} + \epsilon_{gG} \right) 
+ \epsilon_{gQG}\left( \epsilon_{qQ} + \epsilon_{qG} \right) 
+ \epsilon_{gQG}\epsilon_{gQG}
\end{align}
and the significance is then given by 
\begin{equation}
S = N_S \sum_i{ \dfrac{ f_{{qq}_i} p_{qq} + f_{{qg}_i}p_{qg} + f_{{gg}_i}p_{gg}  } {\sqrt{ B_{{qq}_i}p_{qq} + B_{{qg}_i}p_{qg} + B_{{gg}_i}p_{gg}  }}}.
\end{equation}

Exploring selection efficiencies from 30 to 100\% for both quarks and gluons shows that no benefit 
is obtained by applying a quark selection. A small but significant improvement in significance is obtained 
if one of the two jets is required to pass a gluon selection. The resulting significances are  in Fig.~\ref{fig:QuarkGluonSignalSignificance}
and show that the significance increases by about 25\% at high masses (above 5\,\TeV ) with the greatest 
increases  occurring for gluon selection efficiency over 70\%. 


\begin{figure}[htb]
 \centering
\includegraphics[width=0.75\textwidth]{figures/tagging/QuarkGluonSignalSignificance3.png}
\caption{ The significance for observing a \qstar\ with masses from 2000 to 7000\,\GeV\ 
for $\epsilon_{gG}$ ranging from 30 to 90\% compared to the significance calculated with no gluon selection applied. The key gives pairs of efficiencies ($\epsilon_{gG}$, $\epsilon_{qG}$).
  \label{fig:QuarkGluonSignalSignificance}}
\end{figure}

\clearpage


\subsection{Selection Criteria}


In Ref.~\cite{ATL-PHYS-PUB-2017-009}  the selection criteria for  an
enriched quark-initiated jet sample was chosen so that each \pT\ bin had
60\% quark-initiated purity. Applying this criteria to the high mass
dijet sample would lead to discontinuities in the mass spectrum that
would present difficulties to a resonance search. 

Using a selection criteria that is a linear function of the \( \ln(\pT) \) 
results in a smooth mass distribution and can be chosen to produce a  
selection efficiency that is approximately uniform. 
A jet is classed as being more likely to be quark-initiated if \ntrk is less than
the threshold \nq and a more likely to be gluon-initiated if \ntrk is 
greater than the threshold \ngluon  
\begin{align}
\ntrk & \le \nq \; \mbox{quark-initiated sample} \label{eq:QGselect} \\
\ntrk	  & \ge \ngluon \; \mbox{gluon-initiated sample} \nonumber
\end{align}
where   
\begin{equation}
n_{\mathrm{q(g)}} = {c + m \ln(\pT)}  \label{eq:nqg2}
\end{equation}
where $m$ and $c$ are constants chosen to provide suitable subsamples.

The constants $m$ and $c$ are found by finding the value of \ntrk\ 
that corresponds to a given efficiency for truth quark and gluon jets in 
\pT\ bins and fitting the results. For each \pT\ bin the number of tracks 
closest to the chosen selection efficiency is found. Since this is an integer 
number of tracks and does and does not correspond exactly to the selection efficiency 
a correction is applied by estimating the fractional number of tracks that corresponds 
to the selection  efficiency  by carrying out a linear interpolation between the efficiencies 
for the selected bin and its nearest neighbour. The uncertainty on this value is then estimated using 
binomial uncertainties. 


The jet \pT\ bin edges are chosen to be 
480, 500, 520, 540, 560, 580, 600, 625, 650, 700, 750, 800, 900, 1000, 1400, 
1600, 1800, 2000, 2500, 3000, 3500, 4000, 5000, 6000\,\GeV. An example of the \ntrk cumulative 
distribution for truth quark and gluon jets satisfying $800 < \pT < 900\,\GeV$ is shown in
Fig.~\ref{fig:ntrk_cumulative}.


\begin{figure}[htb]
 \centering
\includegraphics[width=0.75\textwidth]{figures/tagging/Cumulative_ntrk_distribution_12_800_900GeV.pdf}
\caption{The cumulative distribution of \ntrk\ for truth quark (blue) and gluon (red) initiated jets 
satisfying $800 < \pT < 900\,\GeV$.  \label{fig:ntrk_cumulative}}
\end{figure}


The constants for Eq.~\ref{eq:nqg2} are found for quark and gluon selection efficiencies from 
65\% to 95\% in 5\% steps. The plot of the value of \ntrk\ that satisfies the selection efficiencies 
of 70, 75 and 80\% are shown in Fig.~\ref{fig:qg_selection_curves} along with the best fit using Eq.~\ref{eq:nqg2}.
The values of the constants for both quark and gluon selections are summarised in 
Tables~\ref{table:truthQuarkSelectionEfficiencies} and \ref{table:truthGluonSelectionEfficiencies}. 
The fit for a selection efficiency of 75\% has a $\chi^2$ of $33.5$ (quark-selection) and $2.6$ 
(gluon-selection) for 21 degrees of freedom. 

The value of \ntrk\ that satisfies the selection efficiency plateaus above 4000\,\GeV\ indicating a possible saturation effect. As a cross check the data is fitted to an alternative fit function 
\begin{equation}
n_{\mathrm{q(g)}} = {c + m \ln(\pT) + n \sqrt{\ln(\pT)}}. \label{eq:nqg3}
\end{equation}
This improves the $\chi^2$ of the fit for a selection efficiency of 75\% for quark-selection from
$33.5$ to $25.1$ and for gluon-selection from $2.6$ to $1.6$.  The improved fit is shown in Fig.~\ref{fig:qg_selection_curves2} and the values of the constants for both quark and gluon selections 
are summarised in 
Tables~\ref{table:truthQuarkSelectionEfficiencies2} and \ref{table:truthGluonSelectionEfficiencies2}. 

\clearpage


\begin{table}[h]
	\centering 
		\caption{ Values of constants $m$ and $c$ from Eq.~\ref{eq:nqg2} such that $ \ntrk  \le \nq $ 
		for truth quark jets for a range of efficiencies  from 65 to 95\%. 
		\label{table:truthQuarkSelectionEfficiencies}
		}
	\begin{tabular}{SSSS}
	\toprule
\multicolumn{1}{c}{Truth-$q$ selection efficiency}   & \multicolumn{1}{c}{Truth-$g$ selection efficiency} &  \multicolumn{1}{c}{$c$}  &  \multicolumn{1}{c}{$m$} \\
\midrule 
0.95 & 0.732 & -27.568 & 8.789 \\
0.90 & 0.563 & -21.518 & 7.269 \\
0.85 & 0.447 & -17.646 & 6.304 \\
0.80 & 0.350 & -14.956 & 5.610 \\
0.75 & 0.278 & -12.600 & 5.022 \\
0.70 & 0.221 & -10.691 & 4.536 \\
0.65 & 0.174 & -8.990 & 4.105 \\
\bottomrule
\end{tabular}
\end{table}

\begin{table}[h]
	\centering 
		\caption{ Values of constants $m$ and $c$ from Eq.~\ref{eq:nqg2} such that $ \ntrk  \ge \ngluon $ 
		for truth quark jets for a range of efficiencies  from 65 to 95\%. 
		\label{table:truthGluonSelectionEfficiencies}
		}
	\begin{tabular}{SSSS}
	\toprule
\multicolumn{1}{c}{Truth-$g$ selection efficiency}   & \multicolumn{1}{c}{Truth-$q$ selection efficiency} &  \multicolumn{1}{c}{$c$}  &  \multicolumn{1}{c}{$m$} \\
\midrule 
 0.95 & 0.586 & -7.541 & 3.233 \\
0.90 & 0.456 & -8.980 & 3.779 \\
0.85 & 0.377 & -10.419 & 4.230 \\
0.80 & 0.320 & -11.964 & 4.659 \\
0.75 & 0.274 & -13.376 & 5.047 \\
0.70 & 0.234 & -14.937 & 5.446 \\
0.65 & 0.202 & -16.466 & 5.834 \\
\bottomrule
\end{tabular}
\end{table}


\begin{table}[h]
	\centering 
		\caption{ Values of constants $m$ and $c$ from Eq.~\ref{eq:nqg3} such that $ \ntrk  \le \nq $ 
		for truth quark jets for a range of efficiencies  from 70 to 80\%. 
		\label{table:truthQuarkSelectionEfficiencies2}
		}
	\begin{tabular}{SSSSS}
	\toprule
\multicolumn{1}{c}{Truth-$q$ selection efficiency}   & \multicolumn{1}{c}{Truth-$g$ selection efficiency} &  \multicolumn{1}{c}{$c$}  &  \multicolumn{1}{c}{$m$} &  \multicolumn{1}{c}{$n$} \\
\midrule 
0.80 & 0.350 & -139.822 & -11.714 & 93.100 \\
0.75 & 0.278 & -128.174 & -11.001 & 86.141 \\
0.70 & 0.221 & -128.255 & -11.755 & 87.604 \\
\bottomrule
\end{tabular}
\end{table}


\begin{table}[h]
	\centering 
		\caption{ Values of constants $m$ and $c$ from Eq.~\ref{eq:nqg3} such that $ \ntrk  \ge \ngluon $ 
		for truth quark jets for a range of efficiencies  from 70 to 80\%. 
		\label{table:truthGluonSelectionEfficiencies2}
		}
	\begin{tabular}{SSSSS}
	\toprule
\multicolumn{1}{c}{Truth-$g$ selection efficiency}   & \multicolumn{1}{c}{Truth-$q$ selection efficiency} &  \multicolumn{1}{c}{$c$}  &  \multicolumn{1}{c}{$m$} &  \multicolumn{1}{c}{$n$}  \\
\midrule 
0.80 & 0.320 & -99.796 & -7.839 & 66.301 \\
0.75 & 0.274 & -99.949 & -7.271 & 65.347 \\
0.70 & 0.234 & -99.774 & -6.640 & 64.077 \\
\bottomrule
\end{tabular}
\end{table}




\begin{figure}[p]
 \centering
  \subfigure[] {\includegraphics[width=0.60\textwidth]{figures/tagging/quark_frac_selection_errors_6}}
  \subfigure[] {\includegraphics[width=0.60\textwidth]{figures/tagging/quark_frac_selection_errors_5}}
  \subfigure[] {\includegraphics[width=0.60\textwidth]{figures/tagging/quark_frac_selection_errors_4}}

\caption{ The values of \ntrk\ for (a) 70\%,  (b) 75\%  and (c) 80\% quark (blue) and gluon (red) 
selection efficiencies in each \pT\ bin along with the best fit to Eq.~\ref{eq:nqg2}.
 \label{fig:qg_selection_curves2}}
\end{figure}

\begin{figure}[p]
 \centering
  \subfigure[] {\includegraphics[width=0.60\textwidth]{figures/tagging/quark_frac_newFit_selection_errors_6}}
  \subfigure[] {\includegraphics[width=0.60\textwidth]{figures/tagging/quark_frac_newFit_selection_errors_5}}
  \subfigure[] {\includegraphics[width=0.60\textwidth]{figures/tagging/quark_frac_newFit_selection_errors_4}}

\caption{ The values of \ntrk\ for  (a) 70\%,  (b) 75\%  and (c) 80\% quark  quark (blue) and gluon (red) 
selection efficiencies in each \pT\ bin along with the best fit to Eq.~\ref{eq:nqg3}.
 \label{fig:qg_selection_curves}}
\end{figure}


\subsubsection{Signal Selection Efficiencies}


The selection criteria for a single jet gluon selection efficiency of 75\% is applied to the \Hprime\ 
sample described in in section~\ref{sec:hprime} are given in Table~\ref{table:HprimeselctionEfficiency}. 
If the selection works perfectly the expected selection efficiency for the \Hprime\ sample would be 56.3\% ($0.75^2$). 
The actual selection efficiencies range between 51.9\% for a 2\,\TeV\ signal to 57.4\% for a 7\,\TeV\ signal. 

The actual fraction of \Hprime\ events that decay to two gluons is less than 100\% due to gluon splitting and other showering 
effects and ranges from 91.3 to 95.4\%. The variation between the expected efficiency (56.3\% of the truth efficiency) 
is plotted in Fig.~\ref{fig:HPrime_efficiency_difference}. The average difference for is approximately 3.3\% for both selection criteria.

There is negligible difference between the two selection criteria so the simpler choice as given in Eq.~\ref{eq:nqg2} will be used.



\begin{table}[h]
	\centering 
		\caption{ The signal selection efficiency for a fully simulated \Hprime\ decaying to two gluons with requiring two jets to 
		pass the 75\% single jet criteria given in Eq.~\ref{eq:nqg2} with constants from 
		Table~\ref{table:truthGluonSelectionEfficiencies} and the criteria given in Eq.~\ref{eq:nqg3} with constants from 
		Table~\ref{table:truthGluonSelectionEfficiencies2}.  
		The expected double tagged gluon efficiency is 56.3\%. 
		\label{table:HprimeselctionEfficiency}
		}
	\begin{tabular}{SSSS}
	\toprule
\multicolumn{1}{c}{\Hprime\ Mass (\GeV)}   & \multicolumn{3}{c}{Selection efficiency(\%)} \\
\multicolumn{1}{c}{} & \multicolumn{1}{c}{Eq.~\ref{eq:nqg2}} & \multicolumn{1}{c}{Eq.~\ref{eq:nqg3} ($\sqrt{}$ term)} 
& \multicolumn{1}{c}{Truth} \\
\midrule 
2000	&	51.9 & 51.8& 91.3 \\
2500	&	53.2 & 53.0& 91.7 \\
3000 	&	54.9 & 54.6& 92.3 \\
3500	&	55.3 & 55.1& 93.4 \\
4000	&	56.4 & 56.2& 93.4 \\
4500	&	56.7 & 56.7& 94.1 \\
5000	&	56.2 & 56.4& 94.3 \\
5500	&	57.2 & 57.5& 94.9 \\
6000	&	57.4 & 57.8& 95.1 \\
6500	&	57.4 & 58.3& 95.5 \\
7000	&	57.4 & 58.1& 95.4 \\\bottomrule
\end{tabular}
\end{table}


\begin{figure}[p]
 \centering
 {\includegraphics[width=0.60\textwidth]{figures/tagging/HPrime_Efficiency_Difference}}

\caption{ The difference between the expected signal selection efficiency of 56.3\%
for a single ject selection efficiency of 75\% for \Hprime\ using  Eq.~\ref{eq:nqg2} (Blue) and
Eq.~\ref{eq:nqg3} (Green)
 \label{fig:HPrime_efficiency_difference}}
\end{figure}



\clearpage

