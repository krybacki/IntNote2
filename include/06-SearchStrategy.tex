\label{sec:searchstrategy}

\subsection{Search strategy for the resonance analysis: the BumpHunter algorithm}

The main statistical test employed in the dijet resonance search is based
on the \BumpHunter\ algorithm~\cite{Aaltonen:2008vt,Choudalakis:2011bh} and 
is used to establish the presence or absence of a resonance in the dijet
mass spectrum, as described in greater detail in previous 
publications~\cite{EXOT-2010-07,EXOT-2011-07}. 
The algorithm operates on the binned \mjj data, comparing the background estimate with the data in mass intervals of 
varying widths formed by combining neighboring bins. Starting with a two-bin signal window,
the algorithm scans across the entire distribution, 
then steps through successively larger signal windows up to half of the whole fit range. 
For each point in the scan, it computes the significance of the difference between the data and the background.
The most significant departure from the smooth spectrum
(``bump'') is defined by the set of bins that have the smallest probability
of arising from a Poisson background fluctuation.
During this procedure, the background model is not changed or refit to the data
outside of the excluded region.

The \BumpHunter\ algorithm accounts for the so-called ``look-elsewhere effect'' 
~\cite{lyons2008, Gross2010}, by performing a series of pseudo-experiments
drawn from the background estimate to determine the probability that random
fluctuations in the background-only hypothesis would create an excess 
anywhere in the spectrum at least as significant as the one observed.

To make practical use of this algorithm, one must ensure the background
estimate is not biased by the signal.
Therefore, \BumpHunter\ is run in two steps.  In the first, the full
distribution is fit and passed to \BumpHunter.
If the most significant local excess from the background fit has a $p$-value
smaller than 0.01, this region is excluded and a new background fit is
performed. The exclusion is then progressively widened bin by bin until
the $p$-value of the remaining fitted region is acceptable.  
During the 2015 analysis it was found that simply excluding one additional bin on the low mass
side of the signal window removed most of the residual bias (if any
did exist).  Then the result of this fit is used for the second
stage where an unbiased estimate of the global significance of any
excess is obtained.
