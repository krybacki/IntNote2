A common, baseline selection is applied for the resonant  analyses of
the Run~2 data that is taken directly from ~\cite{Nishu:2646455}, as
described in Section~\ref{sec:base_selection}. 

%Then, specific cuts applied
%for the resonant analysis, tailored to improve the sensitivity, are described in Sections~\ref{sec:res_selection}. 

\subsection{Observables and Kinematic Variables}
\label{sec:observables}
In the Standard Model (SM), the main and predominant source of dijet
events is two-to-two scattering in QCD processes. The search exploits
two key properties of the background to enhance its sensitivity to new
physics signals:

\begin{itemize}
	\item The background at high \mjj\ is a smooth and continuously falling spectrum.
	\item The background at high mass strongly peaks in the forward
direction, due to Rutherford $t$- and $u$-channel poles in the cross sections
for many individual scattering diagrams \cite{Harris:2011bh}.
\end{itemize}

Many of the possible signals for the dijet search would appear as a
resonance peak in \mjj, the invariant mass of the dijet system formed
from the two highest-\pT\ jets. Resonant signals have $\cos{\theta}$
distributions which, in contrast to Rutherford scattering, are either
isotropic or follow some polynomial in $\cos{\theta}$~\footnote{See
Ref.~\cite{Harris:2011bh} p15 for a summary.}, giving them a distinct
angular distribution from QCD processes. For this reason, the \mjj\
search selects events with small angle separation via an upper limit on
\begin{itemize}
	\item \ystar = $(y_1-y_2)/2$
\end{itemize}
to enhance sensitivity to higher energies and enable probing the scale
of new phenomena. Where $y_{1,2}$ is the rapidity of the leading and
sub-leading jet. The value of the \ystar\ cut is optimized to reject
$t$-channel QCD while admitting the other types of signals.

\subsection{Jet Reconstruction and Calibration}
\label{sec:jet_reconstruction}
Jets are reconstructed with the \akt algorithm \cite{Cacciari:2008gp}
with a distance parameter of 0.4, as implemented in the \textsc{FastJet}
package~\cite{Cacciari:2011ma}. For these studies, we use jets
reconstructed from topological clusters using the procedures described
in \cite{ATLAS-CONF-2015-002}. For jet cleaning, the standard
\textit{Loose} jet quality cuts are used as discussed in
\cite{ATLAS-CONF-2015-029}. An event-based jet cleaning is applied to
reject any event with a leading or sub-leading jet flagged as
\textit{Bad} for the \textit{Loose} criteria.  The jet criteria used is
summarized in Table~\ref{tab:jetCalibration}.

\begin{table}[ht]
	\label{tab:jetCalibration}
		\begin{tabular}{lc}
			\toprule
			%\large
			\multicolumn{2}{c}{Jet reconstruction parameters} \\
			%\normalsize
			%\normalsize
			\midrule
			Parameter & Value \\
			\midrule
			Algorithm & \akt  \\
			R-parameter & 0.4 \\
			Input Constituent & EMTopo \\
			Analysis Release Number & 21.2.121 \\
			%Calibration tag & JetCalibTools-00-04-76 \\
			CalibArea tag & 00-04-82 \\
			%Calibration configuration & JES\_data2017\_2016\_2015\_Consolidated\_EMTopo\_2018\_rel21.config \\
			Calibration configuration & JES\_MC16Recommendation\_Consolidated\_EMTopo\_Apr2019\_Rel21.config \\
			Calibration sequence (Data) & JetArea\_Residual\_EtaJES\_GSC\_Insitu \\
			Calibration sequence (MC) & JetArea\_Residual\_EtaJES\_GSC\_Smear \\
			%Calibration configuration (AFII) & JES\_MC16Recommendation\_AFII\_EMTopo\_April2018\_rel21.config \\
			Calibration configuration (AFII) & JES\_MC16Recommendation\_AFII\_EMTopo\_Apr2019\_Rel21.config \\
			Calibration sequence (AFII) & JetArea\_Residual\_EtaJES\_GSC\_Smear \\
			\midrule
			%\large
			\multicolumn{2}{c}{Selection requirements} \\
			%\normalsize
			\midrule
			Observable & Requirement \\
			\midrule
			Jet cleaning & Loose WP \\
			Batman cleaning & No \\
			\pT  & $>$150 GeV \\
			\textbar$\eta$\textbar & $<$2.1 \\
			\bottomrule
	\end{tabular}
\caption{Jet selection criteria used in this analysis.}
\end{table}


\subsection{Trigger}
\label{sec:trigger}
The data used in this analysis are collected using the lowest
unprescaled small-R single-jet trigger.  The naming convention of
single-jet triggers follows either `Jnnn' for L1 triggers or `jnnn' for
HLT trigger, where 'nnn' is a number specifying the nominal jet \pT
threshold for the trigger in \GeV\ .  The energy scale of the trigger
threshold is the EM scale for L1 triggers, while a calibration sequence
very close to what is applied to offline jets is applied to HLT jets,
bringing their scale to the hadronic scale.

The lowest single jet un-prescaled trigger in 2015 was HLT\_j360, in
2016 was HLT\_j380, in 2017 was HLT\_j420 (if we ignore one run with
luminosity around 2\,\ipb ) and in 2018 was also HLT\_j420. So to keep
the selection consistent across years, the kinetic requirement derived
for HLT\_j420 is used for the analysis. In addition to this,
HLT\_j225\_gsc420\_boffperf\_split is also unprescaled for all data
taking and also considered for the trigger studies for the analysis.

For the complete \RunTwo dataset, the two single-jet unprescaled
triggers for all data taking are: HLT\_j420 and
HLT\_j225\_gsc420\_boffperf\_split as mentioned above, both of which are
seeded from the L1\_J100 trigger. Both triggers search for jets with
$\pT > 420 \GeV$, while the HLT\_j225\_gsc420\_boffperf\_split trigger
also applies the offline global sequential calibration (GSC) to improve
the trigger turn-on.\\

The two triggers were found to be very similar in
performance in Ref.~\cite{Nishu:2646455}, which finally used HLT\_j420 trigger.
It was also found in Ref.~\cite{Nishu:2646455} that obtaining the turn-on directly from \mjj\ provides a much
lower turn-on than from requiring a specific cut on the leading jet \pT.
In order not to sculpt the \mjj\ distribution in an odd way due to
different triggers and given the assessment of the turn-on, this analysis also
uses the HLT\_j420 trigger with no explicit cut on the leading jet \pT.
In practice, the invariant mass cut imposes a soft cut of 380~\GeV on 
the leading jet, and 150~\GeV on the subleading jet. 

\subsection{Baseline selection}
\label{sec:base_selection}
The baseline event selection is applied for all signal regions and
the cuts applied are:
\begin{itemize}
\item Good Run List (GRL): Requirement that all relevant detectors were in a good state ready for physics. The GRLs used for this analysis are:
        \begin{itemize}
                %\item 2015(3.2\,\ifb): data15\_13TeV.periodAllYear\_DetStatus-v89-pro21-02\_Unknown\_PHYS\_ \newline StandardGRL\_All\_Good\,\_25ns.xml
                %\item 2016(32.97\,\ifb): data16\_13TeV.periodAllYear\_DetStatus-v89-pro21-01\_DQDefects-00-02-04\_ \newline PHYS\_StandardGRL\_All\_Good\_25ns.xml
                %\item 2017(44.31\,\ifb): data17\_13TeV.periodAllYear\_DetStatus-v99-pro22-01\_Unknown\_ \newline PHYS\_StandardGRL\_All\_Good\_25ns\_Triggerno17e33prim.xml
                %\item 2018(59.94\,\ifb): data18\_13TeV.periodAllYear\_DetStatus-v102-pro22-04\_Unknown\_PHYS\_ \newline StandardGRL\_All\_Good\_25ns\_Triggerno17e33prim.xml
                \item 2015(3.2\,\ifb): data15\_13TeV.periodAllYear\_DetStatus-v89-pro21-02\_Unknown\_PHYS\\\_StandardGRL\_All\_Good\_25ns.xml
                \item 2016(33\,\ifb): data16\_13TeV.periodAllYear\_DetStatus-v89-pro21-01\_DQDefects-00-02-04\_PHYS\\\_StandardGRL\_All\_Good\_25ns.xml
                \item 2017(44.2\,\ifb): data17\_13TeV.periodAllYear\_DetStatus-v99-pro22-01\_Unknown\_PHYS\\\_StandardGRL\_All\_Good\_25ns\_JetHLT\_Normal2017.xml
                \item 2018(58.5\,\ifb): data18\_13TeV.periodAllYear\_DetStatus-v102-pro22-04\_Unknown\_PHYS\\\_StandardGRL\_All\_Good\_25ns\_Triggerno17e33prim.xml
        \end{itemize}
\item LAr: Liquid Argon Calorimeter error rejected ( errorState(xAOD::EventInfo::LAr) )
\item Tile: Tile Calorimeter error rejected ( errorState(xAOD::EventInfo::Tile) )
\item SCT: SCT single event upsets rejected ( errorState(xAOD::EventInfo::SCT) )
\item Core: Incomplete event build rejected ( isEventFlagBitSet(xAOD::EventInfo::Core, 18) )
\item All jets with $\pt\ge 150\,\GeV$ pass LooseBad cleaning cuts
\item Primary Vertex: the highest $\sum\pt^{2}(trk)$ (xAOD::VxType::VertexType::PriVtx) vertex has at least two tracks associated with it
\item Trigger: Passes the lowest unprescaled single-jet trigger, HLT\_j420
\item Jet preselecton: Leading jet $\pt\ge 380\,\GeV$ and Jet multiplicity $\ge 2$
\item $|\Delta\phi|$ between 2 jets: $|\Delta\phi| > 1.0$
\end{itemize}

Additional kinematic selection criteria are used for various search distributions separately to optimize the
search potential, which are then discussed in Section~\ref{section:ystarCutOptimization}.


\begin{comment}
\subsection{Analysis cutflow}
%\label{sec:data_cutflow}

This section and the next present the analysis cutflows. Cutflows obtained on
Run-2 data are presented in Tables~\ref{tab:cutFlow_resonance_run2} and 
\ref{tab:cutFlow_wstar_run2}.

\begin{table}[htbp]
	\centering
	\begin{tabular}{l|c|c}
		\hline\hline
		Selection criteria & $N_{events}$ & rel. decrease (\%) \\
		\hline
		all      &	4738142726	&	0.00	\\
		Apply GRL 	& 	4442605390        & 	-6.24	 \\
		Cleaning	 & 	4379077017	 & 	-1.43	 \\
		HLT j420	 & 	266104885	 & 	-93.9	 \\
		jet pre-selection	 &     259157844         &      -2.61    \\
		$|\Delta\phi| > 1.0$	 & 		 & 		 \\
		$|\ystar| < 0.6$	 & 		 & 		 \\
		$\mjj>1100~\GeV$	 & 		 & 		 \\
		\hline\hline
	\end{tabular}
	\caption{Cutflow for
		events with H$^\prime$ cuts:  $\mjj>1100~\GeV$, and $|\ystar|<0.6$. .
		\label{tab:cutFlow_resonance_run2} }
\end{table}

\begin{table}[htbp]
	\centering
	\begin{tabular}{l|c|c}
		\hline\hline
		Selection criteria & $N_{events}$ & rel. decrease (\%) \\
		\hline
		all      &	4738142726	&	0.00	\\
		Apply GRL 	& 	4442605390        & 	-6.24	 \\
		Cleaning	 & 	4379077017	 & 	-1.43	 \\
		HLT j420	 & 	266104885	 & 	-93.9	 \\
		jet pre-selection	 &     259157844        &      -2.61    \\
		$|\Delta\phi| > 1.0$	 & 		 & 		 \\
		$|\ystar| < 0.8$	 & 		 & 	 \\
		$\mjj>1133~\GeV$	 & 		 & 	 \\
		\hline\hline
	\end{tabular}
	\caption{Cutflow for
		events with string resonance cuts:  $\mjj>1133~\GeV$, and $|\ystar|<0.8$. .
		\label{tab:cutFlow_wstar_run2} }
\end{table}
\end{comment}
