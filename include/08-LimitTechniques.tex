\label{sec:limitsetting}

The limit setting phase is performed in case no significant deviations from the
background-only hypothesis are reported in the search phase. While the search
phase only depends on the BKG
distribution and its (dis)agreement with data, the limit setting phase performs
exclusion limits given certain theory models. In the following a more detailed
description on the statistical machinery used to set limits will be given.

\subsection{Limit setting using the dijet mass distribution}

Exclusion limits were calculated using the CL$_\mathrm{s}$ method with a binned profile likelihood ratio as the test
statistic using the \texttt{HistFitter} \cite{HistFitter:2014} framework.
The limit is extracted from adjusting the normalization $\mu_{s}$ of a template that represents the shape of the nominal signal model in question.
By scanning over values of $\mu_{s}$ until a CL$_\mathrm{s}$ p-value of 0.05 is found,
a 95\% confidence upper limit on this normalization parameter can be found.
The obtained upper limit on signal strength is then converted into an upper limit of cross section
corresponds to given luminosity and the observed data.
The limit on $\sigma\times {\cal A} \times BR$ from data is interpolated logarithmically
between mass points to create a continuous curve in \mjj{}.
The exclusion limit on the mass (or energy scale) of the
given new physics resonant signal occurs at the value of \mjj{} where the limit
on $\sigma\times {\cal A} \times BR$ from data is the same as the
theoretical value, which is derived from the interpolation between
the generated mass values.

Expected limits on the signal model are calculated by using an asymptotic approximation. A major benefit
of using a binned likelihood approach such as used with this \texttt{HistFitter} approach is it's
computational speed, particularly due to the asymptotic approximation.
The \texttt{HistFitter} method has calculated expected and observed limit for each mass point around 30 seconds.
This is compared to hours or days using the Bayesian approach for very similar results.
%
%
%\subsection{Limit setting using the dijet mass distribution}
%
%For each benchmark process under study in the resonance analysis, Monte Carlo samples have been
%simulated at a number of selected mass points, $m_{\textrm{R}}$.
%The Bayesian method documented in ref.~\cite{Aad:2011aj}
%is applied to data at these same mass points to set a
%95\% CL limit on the cross section times acceptance times branching ratio,
%$\sigma\times {\cal A} \times BR$, for the new physics resonant signal as a function of
%$m_{\textrm{R}}$, using a prior constant in signal strength.  The one exception is QBH, which has limits set on $\sigma\times {\cal A}$.
%The limit on $\sigma\times {\cal A} \times BR$ from data is interpolated logarithmically
%between mass points to create a continuous curve in \mjj{}.
%The exclusion limit on the mass (or energy scale) of the
%given new physics resonant signal occurs at the value of \mjj{} where the limit
%on $\sigma\times {\cal A} \times BR$ from data is the same as the
%theoretical value, which is derived from the interpolation between
%the generated mass values.

This form of analysis is applicable to all resonant phenomena
where the new physics resonance couplings are strong compared to the scale of perturbative QCD
at the signal mass, so that interference with QCD terms can be neglected.
The acceptance calculation includes all reconstruction steps and analysis cuts is
described in Section~\ref{sec:res_selection}.

The background prediction used in this calculation is obtained from a
simultaneous S+B fit.

The effects of the systematic uncertainties on the data-driven background together with the systematic uncertainties
on the signal samples, described in Section~\ref{sec:systematicUncertainties}, are considered in the limit-setting procedure.
These uncertainties are incorporated into the LH formalism by varying all sources according
to Gaussian probability distributions.
%The templates are linearly interpolated
%between 1$\sigma$ and -1$\sigma$ and exptrapolated to reach 3$\sigma$ with an exponential function
%for points above that interval according to the recomandation of the stat.forum.
Confidence intervals are then calculated from the resulting profile of
the parameter-of-interest of the LH.

In order to validate the limit setting machinery, results have been compared
with
the Bayesian limits used in previous iteration of the analysis. In addition, a
validation of the asymptotic approximation has been checked comparing limits
derived from toys. Both studies are described in Appendix~\ref{sec:HF_val}.
